% !TEX TS-program = pdflatex
% !TEX encoding = UTF-8 Unicode
\documentclass[a4paper,twoside,11pt]{article}
\usepackage[polish]{babel}
\usepackage[MeX]{polski}
\usepackage[utf8]{inputenc}
\usepackage{indentfirst,graphicx,amsmath}
\usepackage[left=2cm,top=2cm,right=2cm,nohead]{geometry}
\usepackage{hyperref}
\usepackage{tabularx}
\usepackage{color}
\usepackage{float}
\frenchspacing
\pagestyle{plain}

\makeatletter
\renewcommand\@seccntformat[1]{\csname the#1\endcsname.\quad}
\renewcommand\numberline[1]{#1.\hskip0.7em}
\makeatother

\newenvironment{mylisting}
{\begin{list}{}{\setlength{\leftmargin}{1em}}\item\scriptsize\bfseries}
{\end{list}}

\newenvironment{mytinylisting}
{\begin{list}{}{\setlength{\leftmargin}{1em}}\item\tiny\bfseries}
{\end{list}}

\usepackage{graphicx}
\usepackage{color}

\author{\\ ~ \\ Tomasz Gański \\
Tomasz Gieniusz \\
Bartosz Jankowski\\
Grzegorz Marcinkowski\\
Łukasz Odzioba\\
Jacek Weremko}
\title{\LARGE Wykrywacz zachcianek gastronomicznych}

\begin{document}

\makeatletter

\renewcommand{\maketitle}{
	\begin{titlepage}
		\begin{center}
			\small Politechnika Gdańska \\ Wydział Elektroniki, Telekomunikacji i Informatyki
		\end{center}
		\vspace{3cm}
		\noindent \rule{\linewidth}{0.4mm}
		\begin{center}
			\textsc{
				\huge{Projekt systemu } \\
				\vspace{0.3ex}
				\LARGE{Systemy z Bazą Wiedzy}
			}
			\\
			\vspace{1em}
			\Large{\@title}
		\end{center}
		\rule{\linewidth}{0.4mm}
		\vspace{1em}
		\begin{center}
			\large Autorzy: \@author
		\end{center}
		\vspace{7em}
		\ifpdf
			\begin{tabularx}{\textwidth}{ X X }
				\makebox[0.5\textwidth][l]{
					\hspace{1cm}
					\includegraphics[width=4cm]{logo_pg}
				}
				&
				\makebox[0.5\textwidth][r]{
					\includegraphics[width=4cm]{logo_eti}
					\hspace{1cm}
				}
			\end{tabularx}
		\fi
		% Duży odstęp, który powoduje, że stopka z datą pojawia się na dole. Dokładniej wypycha ją
		% poza stronę, ale latex na to nie pozwala i umieszcza ją na końcu strony.
		\vspace*{\stretch{6}}
		\begin{center}
			Gdańsk, \@date \ r.
		\end{center}
	\end{titlepage}
}

\makeatother

\maketitle{}


\tableofcontents

\newpage

\section{Cel projektu}

\section{Grupa docelowa}

\section{Wymagania}

\section{Technologia}



\end{document}
